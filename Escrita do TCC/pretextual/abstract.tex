% --- resumo em inglês ---
	%%Altere as informações do resumo%%
\noindent{SILVA; Bruno Prece da. \textbf{\imprimirtitulo}. Trabalho de Conclusão de Curso (Graduação) - \imprimirinstituicao. \imprimirlocal, \imprimirdata.}
\par
	%%Se o seu resumo não for em inglês, altere o ``Abstract'' e ``english'' abaixo.
\begin{resumo}[Abstract]
\begin{otherlanguage*}{english}
	The detection of anomalies in networks is an area of very dynamic research that involves various techniques to detect known and unknown anomalies. Several systems and algorithms are proposed and there are many discussions concerning the effectiveness of each method in the literature. This paper presents some aspects of anomaly detection, and discusses in particular the clustering techniques. A case study using the KDD99 database and the K-means algorithm is presented.
\newline
\newline
\vspace{\onelineskip}
\noindent
\emph{
\noindent\textbf{Keywords}: Anomaly Detection, clustering, information security.}
\end{otherlanguage*}
\end{resumo}

% --- resumo em português ---
	%%Altere as informações do resumo%%
\noindent{SILVA; Bruno Prece da, \textbf{\imprimirtitulo}. Trabalho de Conclusão de Curso (Graduação) - \imprimirinstituicao. \imprimirlocal, \imprimirdata.}
\par
\begin{resumo}
\indent A detecção de anomalias em redes é uma área de pesquisa muito dinâmica, que envolve diversas técnicas para detectar anomalias conhecidas e desconhecidas. Diversos sistemas e algoritmos são propostos e há muitas discussões referentes à eficácia de cada método na literatura. Este trabalho apresenta alguns aspectos da detecção de anomalias, e discute em particular as técnicas de agrupamento. Um estudo de caso utilizando a base de dados do KDD99 e o algoritmo K-means é apresentado.
\vspace{\onelineskip} \\
	%%Adicione as palavras chaves após os dois pontos '':''%%
\noindent
\textbf{Palavras-chaves}: Detecção de Anomalias, clustering, segurança da informação.
\end{resumo}

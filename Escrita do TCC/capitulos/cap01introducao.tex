\chapter{Introdução}%%Inserir título do capítulo (nível 1)
\indent A constante expansão na utilização da Internet tornou a rede um serviço essencial para a interconexão global e para as empresas. A cada dia há um crescimento na demanda por  novos serviços, que necessitam de políticas de segurança que mantenham a integridade e a privacidade dos dados. Isso torna os sistemas cada vez mais complexos e os dados cada vez mais heterogêneos, impossibilitando que a gerência da rede seja realizada por um operador humano. Devido à esta dificuldade de gerência, surgem comportamentos que fogem dos padrões normais do tráfego, estes comportamentos são as anomalias. Uma anomalia pode ser definida como uma não conformidade com o comportamento normal de uma determinada base de dados.

\indent As anomalias podem ser geradas por vários fatores, podendo ser maliciosas, como, por exemplo, ataques de negação de serviço (\textit{Dos}), exércitos de máquinas controladas sem autorização (\textit{botnets}) e envio massivo de correio eletrônico (\textit{spam}). E podem não ser maliciosas, como, interrupções não planejadas, crescimento repentino de tráfego, e tráfego de uma única fonte para muitos destinos. Deste modo, os \textit{firewalls} não são suficientes para manter segurança e integridade dos sistemas, pois, não conseguem evitar ataques externos. Já os sistemas de autenticação, não conseguem evitar um comportamento nocivo de um usuário legítimo, além disso, estes sistemas possuem limitações para verificar anomalias causadas pela utilização não nociva e por problemas nos equipamentos.

\indent A detecção de anomalias em redes é uma tarefa extremamente difícil, principalmente porque as anomalias são alvos móveis presentes em um conjunto de dados heterogêneos, que dificulta a precisão em encontrar um determinado conjunto de dados do tráfego que caracterizam uma anomalia. Além disso, para classificar uma anomalia é necessário conhecer o seu perfil, e a dificuldade desta tarefa aumenta a cada dia com o surgimento de novos tipos ataques e técnicas de invasão.

\indent De maneira geral, os estudos classificam a detecção de anomalias em duas vertentes, baseada no comportamento normal da rede, onde busca-se caracterizar um perfil do tráfego normal, e a cada novo evento é realizada uma comparação com o perfil normal, caso o evento não se enquadre, será considerado um comportamento anômalo. E na detecção baseada em assinaturas, será implementado um banco de dados com o perfil de várias anomalias, e assim, cada evento do tráfego será confrontado com estes perfis anômalos, a fim de verificar se o evento capturado possuí características semelhantes às anomalias.

\indent O objetivo deste trabalho é apresentar os principais métodos de detecção de anomalias presentes na literatura, trazer análises referentes às vantagens e desvantagens de cada abordagem e os tipo de dados em que apresentam maior eficaz. Além disso, apresentamos um estudo de caso utilizando a técnica de clusterização dos dados, uma abordagem simplificada, muito explorada na literatura e apresenta bons resultados.

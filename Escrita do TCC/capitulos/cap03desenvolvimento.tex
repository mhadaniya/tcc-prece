\chapter{Desenvolvimento}

  \section{Bases de Dados}
\indent A base de dados de rede conhecida como DARPA foi criada pelo Lincoln Laboratory do Instituto Tecnológico de Massachusetts para avaliação de sistemas de detecção de intrusão e consequentemente é utilizado para detecção de anomalias. Na DARPA estão contidos dados de tráfego gerado a partir de uma mistura de máquinas reais e máquinas simuladas, onde o tráfego de fundo foi gerado artificialmente por estas máquinas e os ataques foram realizados contra as máquinas reais. Apesar da incerteza em relação a naturalidade do tráfego gerado artificialmente, esta base de dados é utilizada em grande escala na literatura.

\indent Dois cenários de ataques são simulados na DARPA o LLDOS 1.0 e o LLDOS 2.0. O primeiro cenário simula um atacante não muito experiente, que deixa clara sua presença na rede e no segundo cenário é simulado um atacante experiente que não deixa clara a sua presença na rede. Os ataques gerados nos dois cenários se baseiam em scanner de portas, ataques de negação de serviços e instalação de worms na rede.



  \section{Implementação}

  \section{Resultados}
